\documentclass{report}
\usepackage{amsmath}
\usepackage{amsfonts} 
\usepackage{physics} 
\usepackage{rotating}
\usepackage{enumitem}
\setlength{\headheight}{15pt}
\usepackage[italian]{varioref}
\usepackage{hyperref}
\usepackage{steinmetz}
\hypersetup{colorlinks=true, linkcolor=magenta, citecolor=green, urlcolor=blue} 
\usepackage{mathtools}
\hypersetup{linktoc=page}
\usepackage{extramarks}
\usepackage{amssymb}
\usepackage{fancyhdr}
\usepackage{float}
\usepackage{geometry}
 \geometry{
 a4paper,
 total={190mm,257mm},
 left=15mm,
 right=15mm,
 top=20mm,
 }
\usepackage{graphicx}
\usepackage{cancel}
\graphicspath{ {./images/} }
\newcommand{\R}{\mathbb{R}}
\newcommand{\N}{\mathbb{N}}
\usepackage{amsthm}
\usepackage{etoolbox}
\newtheoremstyle{break}
  {\topsep}{\topsep}%
  {\itshape}{}%
  {\bfseries}{}%
  {\newline}{}%
\theoremstyle{break}
\newtheorem{definition}{Definition}[section]
\newtheorem{example}{Example}[section]
\newtheorem{hw}{Exercise}[section]
\newtheorem{theorem}{Theorem}[section]
\BeforeBeginEnvironment{minipage}{\medskip}
\AfterEndEnvironment{minipage}{\medskip}
\usepackage{tocloft}

\usepackage{indentfirst}
\usepackage{tikz}
\usepackage{afterpage}
\numberwithin{equation}{section}
\newcommand\blankpage{%
    \null
    \thispagestyle{empty}%
    \addtocounter{page}{-1}%
    \newpage}
\newcommand\blankpagewnumber{%
    \null
    \newpage}
\newcommand{\T}{\mathcal{T}}
\newcommand{\U}{\mathcal{U}}
\renewcommand{\L}{\mathcal{L}}
\usepackage{blkarray, bigstrut}
\usepackage[T1]{fontenc}
% --- Macro \xvec
\makeatletter
\newlength\xvec@height%
\newlength\xvec@depth%
\newlength\xvec@width%
\newcommand{\xvec}[2][]{%
  \ifmmode%
    \settoheight{\xvec@height}{$#2$}%
    \settodepth{\xvec@depth}{$#2$}%
    \settowidth{\xvec@width}{$#2$}%
  \else%
    \settoheight{\xvec@height}{#2}%
    \settodepth{\xvec@depth}{#2}%
    \settowidth{\xvec@width}{#2}%
  \fi%
  \def\xvec@arg{#1}%
  \def\xvec@dd{:}%
  \def\xvec@d{.}%
  \raisebox{.2ex}{\raisebox{\xvec@height}{\rlap{%
    \kern.05em%  (Because left edge of drawing is at .05em)
    \begin{tikzpicture}[scale=1]
    \pgfsetroundcap
    \draw (.05em,0)--(\xvec@width-.05em,0);
    \draw (\xvec@width-.05em,0)--(\xvec@width-.15em, .075em);
    \draw (\xvec@width-.05em,0)--(\xvec@width-.15em,-.075em);
    \ifx\xvec@arg\xvec@d%
      \fill(\xvec@width*.45,.5ex) circle (.5pt);%
    \else\ifx\xvec@arg\xvec@dd%
      \fill(\xvec@width*.30,.5ex) circle (.5pt);%
      \fill(\xvec@width*.65,.5ex) circle (.5pt);%
    \fi\fi%
    \end{tikzpicture}%
  }}}%
  #2%
}
\makeatother

% --- Override \vec with an invocation of \xvec.
\let\stdvec\vec
\renewcommand{\vec}[1]{\xvec[]{#1}}
% --- Define \dvec and \ddvec for dotted and double-dotted vectors.
\newcommand{\dvec}[1]{\xvec[.]{#1}}
\newcommand{\ddvec}[1]{\xvec[:]{#1}}

\bibliographystyle{plain} 

\usepackage{algpseudocode}
\begin{document}
\afterpage{\blankpage}
\begin{titlepage}
\begin{center}
\huge Master's degree in Computer Engineering for Robotics and Smart Industry
\end{center}
\vspace*{\fill}
\begin{center}
\textbf{\Huge Robotics, Vision and Control}
\end{center}
\begin{center}
Report on the assignments given during the 2021/2022 a.y.
\end{center}
\vspace*{\fill}
\begin{center}
\begin{minipage}{0.4\textwidth}
\begin{flushleft}
Author: Lorenzo Busellato, VR472249\\
email: lorenzo.busellato\_02@studenti.univr.it
\end{flushleft}
\end{minipage}
\begin{minipage}{0.5\textwidth}
\begin{flushright}
\includegraphics[keepaspectratio,width=0.6\textwidth]{logo}
\end{flushright}
\end{minipage}

\end{center}
\vspace{1cm}
\end{titlepage}


\afterpage{\blankpage}
\thispagestyle{empty}
\setcounter{page}{0}
\pagenumbering{gobble}
\renewcommand{\cftsecleader}{\cftdotfill{\cftdotsep}}
\setcounter{tocdepth}{2}
\tableofcontents
\clearpage

\fancyhead[R]{}
\fancyhead[L]{}
\pagestyle{fancy}
\pagenumbering{arabic}

\chapter{Robotics}

\section{Assignment 1}

\subsection{Implement in MATLAB 3rd-, 5th-, 7th-order polynomials for $q_i>q_f$ and $q_i<q_f$ and for $t\in [t_i,t_f]$ and $t\in[0,\Delta T]$}

All polynomial trajectories can be expressed as:

\begin{align*}
q(t) &= a_7(t-t_i)^7 + a_6(t-t_i)^6 + a_5(t-t_i)^5 + a_4(t-t_i)^4 + a_3(t-t_i)^3 + a_2(t-t_i)^2 + a_1(t-t_i) + a_0\\
\dot q(t) &= 7a_7(t-t_i)^6 + 6a_6(t-t_i)^5 + 5a_5(t-t_i)^4 + 4a_4(t-t_i)^3 + 3a_3(t-t_i)^2 + 2a_2(t-t_i) + a_1\\
\ddot q(t) &= 42a_7(t-t_i)^5 + 30a_6(t-t_i)^4+20a_5(t-t_i)^3 + 12a_4(t-t_i)^2 + 6a_3(t-t_i) + 2a_2\\
\dddot q(t) &= 210a_7(t-t_i)^4 + 120a_6(t-t_i)^3+60a_5(t-t_i)^2 + 24a_4(t-t_i) + 6a_3\\
\ddddot q(t) &= 840a_7(t-t_i)^3 + 360a_6(t-t_i)^2+120a_5(t-t_i) + 24a_4
\end{align*}

For the 3rd-order polynomial $a_7=a_6=a_5=a_4=0$ while for the 5th-order polynomial $a_7=a_6=0$.

The problem of determining the $a_i$ coefficients of the polynomials is solved by setting up a system of equations using initial and final conditions on velocity (3rd-order), velocity and acceleration (5th-order), velocity, acceleration and jerk (7th-order).

\begin{figure}[H]
\centering
\includegraphics[keepaspectratio,width=\textwidth]{poly_1}
\caption{3rd-, 5th-, 7th-order polynomial trajectories with $q_i<q_f$ and $t\in[0,\Delta T]$, $q_i=0,q_f=1,\Delta T=1, v_i=v_f=a_i=a_f=j_i=j_f=0$.}
\label{fig:poly_1}
\end{figure}

\begin{figure}[H]
\centering
\includegraphics[keepaspectratio,width=\textwidth]{poly_2}
\caption{3rd-, 5th-, 7th-order polynomial trajectories with $q_i>q_f$ and $t\in[0,\Delta T]$, $q_i=2,q_f=1,\Delta T=1, v_i=v_f=a_i=a_f=j_i=j_f=0$.}
\label{fig:poly_2}
\end{figure}

\begin{figure}[H]
\centering
\includegraphics[keepaspectratio,width=\textwidth]{poly_3}
\caption{3rd-, 5th-, 7th-order polynomial trajectories with $q_i<q_f$ and $t\in[t_i,t_f]$, $q_i=0,q_f=1,t_i=1,t_f=2, v_i=v_f=a_i=a_f=j_i=j_f=0$.}
\label{fig:poly_3}
\end{figure}

\begin{figure}[H]
\centering
\includegraphics[keepaspectratio,width=\textwidth]{poly_4}
\caption{3rd-, 5th-, 7th-order polynomial trajectories with $q_i>q_f$ and $t\in[t_i,t_f]$, $q_i=2,q_f=1,t_i=1,t_f=2, v_i=v_f=a_i=a_f=j_i=j_f=0$.}
\label{fig:poly_4}
\end{figure}

\section{Assignment 2}

\subsection{Implement in MATLAB the trapezoidal trajectory taking into account the different constraints.}

The general expression for a trapezoidal velocity profile is:

\begin{equation*}
q(t)=\begin{cases}
q_i+\dot q_i(t-t_i)+\frac{\dot q_c-\dot q_i}{2t_a}(t-t_i)^2 & t_i\leq t\leq t_a+t_i\\
q_i+\dot q_i\frac{t_a}{2}+\dot q_c(t-t_i-\frac{t_a}{2})^2 & t_a+t_i\leq t\leq t_f-t_d\\
q_f-\dot q_f(t_f-t)-\frac{\dot q_c-\dot q_f}{2t_d}(t_f-t)^2 & t_f-t_d\leq t\leq t_f
\end{cases}
\end{equation*}

If the initial and final velocities are null, then $t_a=t_d=t_c$.

If $\dot q_i=\dot q_f=0$ then the possible constraints are:
\begin{itemize}
\item $t_c$:
\begin{equation*}
\ddot q_c = \frac{q_f-q_i}{t_ct_f-t_c^2}\;\;\;\;\dot q_c=\ddot q_ct_c
\end{equation*}
\item $\ddot q_c$:
\begin{equation*}
t_c=\frac{t_f}{2}-\frac{1}{2}\sqrt{\frac{t_f^2\ddot q_c-4(q_f-q_i)}{\ddot q_c}}\;\;\;\;\dot q_c=\ddot q_ct_c
\end{equation*}
\item $\dot q_c$:
\begin{equation*}
t_c=\frac{q_i-q_f+\dot q_ct_f}{\dot q_c}\;\;\;\;\ddot q_c = \frac{\dot q_c^2}{q_i-q_f+\dot q_ct_f}
\end{equation*}
\item $\ddot q_c,\dot q_c$:
\begin{equation*}
t_c=\frac{\dot q_c}{\ddot q_c}\;\;\;\;t_f=\frac{\dot q_c^2+\ddot q_c(q_f-q_i)}{\dot q_c\ddot q_c}
\end{equation*}
\end{itemize}

In all cases the feasibility condition is that $2t_c\leq t_f-t_i$.

If $\dot q_i,\dot q_f\neq 0$ the possible constraints are:

\begin{itemize}
\item $\ddot q_{c,max}$:
\begin{equation*}
\dot q_c = \frac{1}{2}(\dot q_i+\dot q_f+\ddot q_{c,max}\Delta T+\sqrt{\ddot q_{c,max}^2\Delta T^2-4\ddot q_{c,max}\Delta q+2\ddot q_{c,max}(\dot q_i+\dot q_f)\Delta T-(\dot q_i-\dot q_f)^2}\;\;\;\;\;t_a=\frac{\dot q_c-\dot q_i}{\ddot q_{c,max}}\;t_d=\frac{\dot q_c-\dot q_f}{\ddot q_{c,max}}
\end{equation*}

The trajectory is feasible when the argument of the square root is positive, and when the maximum acceleration satisfies:

\begin{equation*}
\ddot q_{c,max}\Delta q>\frac{\abs{\dot q_i^2-\dot q_f^2}}{2}\;\;\;\ddot q_{c,max}\geq\ddot q_{c,lim}=\frac{2\Delta q-(\dot q_i-\dot q_f)\Delta T+\sqrt{4\Delta q^2-4\Delta q(\dot q_i+\dot q_f)\Delta T+2(\dot q_i^2+\dot q_f^2)\Delta T^2}}{\Delta T^2}
\end{equation*}

when $\ddot q_{c,max}=\ddot q_{c,lim}$ there is no constant velocity phase.
\item $\ddot q_{max},\dot q_{max}$:
First we compute the condition:
\begin{equation*}
\ddot q_{c,max}\Delta q\gtreqless\dot q_{c,max}^2-\frac{\dot q_i^2+\dot q_f^2}{2}
\end{equation*}

If the above is $>$ then:

\begin{equation*}
\dot q_c=\dot q_{c,max}\;\;\;t_a=\frac{\dot q_{c,max}-\dot q_i}{\ddot q_{c,max}}\;\;\;t_d=\frac{\dot q_{c,max}-\dot q_f}{\ddot q_{c,max}}\;\;\;\Delta T=\frac{\Delta q\ddot q_{c,max}+\dot q_{c,max}^2}{\ddot q_{c,max}\dot q_{c,max}}
\end{equation*}

If the above is $\leq$ then:
\begin{equation*}
\dot q_c=\dot q_{c,lim}=\sqrt{\ddot q_{c,max}\Delta q+\frac{\dot q_i^2+\dot q_f^2}{2}}<\dot q_{c,max}\;\;\;t_a=\frac{\dot q_{c,lim}-\dot q_i}{\ddot q_{c,max}}\;\;\;t_d=\frac{\dot q_{c,lim}-\dot q_f}{\ddot q_{c,max}}
\end{equation*}
\end{itemize}

\begin{figure}
\begin{minipage}{0.333\textwidth}
\centering
\includegraphics[keepaspectratio,width=\textwidth]{trap_1}
\caption{$t_c=2s$}
\label{fig:trap_1}
\end{minipage}
\begin{minipage}{0.333\textwidth}
\centering
\includegraphics[keepaspectratio,width=\textwidth]{trap_2}
\caption{$t_c=2s, q_i>q_f$}
\label{fig:trap_2}
\end{minipage}
\begin{minipage}{0.333\textwidth}
\centering
\includegraphics[keepaspectratio,width=\textwidth]{trap_3}
\caption{$t_c=2s, t_i\neq 0$}
\label{fig:trap_3}
\end{minipage}
\end{figure}

\begin{figure}
\begin{minipage}{0.5\textwidth}
\centering
\includegraphics[keepaspectratio,width=\textwidth]{trap_5}
\caption{$v_c=1.5$}
\label{fig:trap_5}
\end{minipage}
\begin{minipage}{0.5\textwidth}
\centering
\includegraphics[keepaspectratio,width=\textwidth]{trap_4}
\caption{$v_c=v_{c,lim}=2$}
\label{fig:trap_4}
\end{minipage}
\end{figure}

\begin{figure}
\begin{minipage}{0.5\textwidth}
\centering
\includegraphics[keepaspectratio,width=\textwidth]{trap_6}
\caption{$a_c=2$}
\label{fig:trap_6}
\end{minipage}
\begin{minipage}{0.5\textwidth}
\centering
\includegraphics[keepaspectratio,width=\textwidth]{trap_7}
\caption{$a_c,v_c=2$}
\label{fig:trap_7}
\end{minipage}
\end{figure}

\begin{figure}
\begin{minipage}{0.5\textwidth}
\centering
\includegraphics[keepaspectratio,width=\textwidth]{trap_8}
\caption{$\dot q_i=1,\dot q_f=0.5,\ddot q_{max}=0.5$}
\label{fig:trap_8}
\end{minipage}
\begin{minipage}{0.5\textwidth}
\centering
\includegraphics[keepaspectratio,width=\textwidth]{trap_9}
\caption{$\dot q_i=1,\dot q_f=0.5,\dot q_{max}=1.5,\ddot q_{max}=0.5$}
\label{fig:trap_9}
\end{minipage}
\end{figure}

To generate a multipoint trajectory we simply repeat the computation for each consecutive point pair. To improve the resulting velocity and acceleration profiles we can apply an heuristic to assign velocities to each waypoint, so that a less jagged profile is obtained:

\begin{align*}
\dot q(t_i)&=\dot q_i\\
\dot q(t_k)&=\begin{cases}
0 & \text{if }sign(\Delta Q_k)\neq sign(\Delta Q_{k+1})\\
sign(\Delta Q_k)\dot q_{max}& \text{if }sign(\Delta Q_k)= sign(\Delta Q_{k+1})\\
\end{cases}\\
\dot q(t_f)&=\dot q_f
\end{align*}

\begin{figure}
\begin{minipage}{0.5\textwidth}
\centering
\includegraphics[keepaspectratio,width=\textwidth]{trap_10}
\caption{Multipoint trajectory without velocity heuristic}
\label{fig:trap_10}
\end{minipage}
\begin{minipage}{0.5\textwidth}
\centering
\includegraphics[keepaspectratio,width=\textwidth]{trap_11}
\caption{Multipoint trajectory with velocity heuristic}
\label{fig:trap_11}
\end{minipage}
\end{figure}

\section{Assignment 3}
\subsection{Interpolating polynomials with computed velocities at path points and imposed velocity at initial/final points.}

To implement multipoint trajectories we concatenate cubic splines. The interpolating trajectory is then:

\begin{equation*}
q(t)\coloneqq\{\Pi_k(t),t\in[t_k,t_{k+1}],k=0,\dots,n-1\}
\end{equation*}

where:

\begin{equation*}
\Pi(t)=a_3^k(t-t_k)^3+a_2^k(t-t_k)^2+a_1^k(t-t_k)+a_0^k
\end{equation*}

Fixing velocities on all path points yields:

\begin{equation*}
\begin{cases}
a_0^k&=q_k\\
a_1^k&=\dot q_k\\
a_2^k&=\frac{1}{T_k}\left(\frac{3(q_{k+1}-q_k)}{T_k}-2\dot q_k-\dot q_{k+1}\right)\\
a_3^k&=\frac{1}{T_k^2}\left(\frac{^2(q_{k}-q_{k+1})}{T_k}+\dot q_k+\dot q_{k+1}\right)\\
\end{cases}
\end{equation*}

where $T_k=t_{k+1}-t_k$.

\begin{figure}[h]
\centering
\includegraphics[keepaspectratio,width=\textwidth]{cubic_1}
\caption{Trajectory through $q_k=\begin{bmatrix}
10 & 20 & 30 & 0 & 40
\end{bmatrix}$ at times $t_k=\begin{bmatrix}
0 & 2 & 4 & 8 & 10
\end{bmatrix}$ with \\velocities $\dot q_k=\begin{bmatrix}
0 & 0 & 0 & 5.2 & 0
\end{bmatrix}$}
\end{figure}

Path point velocities are not generally known. We can estimate them with Euler's approximation:

\begin{equation*}
v_k = \frac{q_k-q_{k-1}}{t_k-t_{k-1}}\implies\begin{cases}
\dot q(t_0)&=\dot q_0\\
\dot q(t_k)&=\begin{cases}
0 & \text{if }sign(v_k)\neq sign(v_{k+1})\\
\frac{v_k+v_{k+1}}{2} &  \text{if }sign(v_k) =  sign(v_{k+1})\\
\end{cases}\\
\dot q(t_n) &= \dot q_n
\end{cases}
\end{equation*}

\begin{figure}[h]
\centering
\includegraphics[keepaspectratio,width=\textwidth]{cubic_2}
\caption{Trajectory interpolation with Euler's approximation.}
\end{figure}

\newpage

\subsection{Interpolating polynomials with continuous accelerations at path points and imposed velocity at initial/final points (+ Thomas algorithm)}


Imposing velocity and acceleration continuity at path points results in the definition of a linear system $A\dot q=c$, where $A$ is a tridiagonal matrix. Thanks to this property the system can be solved for $\dot q$ efficiently using Thomas' algorithm. Given:

\begin{equation*}
\begin{bmatrix}
2(T_0+T_1) & T_0 &  &\\
T_2 & 2(T_1+T_2) & T_1 &  & \\
 & \ddots & \ddots & \ddots &  & \\
& & T_{k+1} & 2(T_k+T_{k+1}) & T_k &  & \\
& & & \ddots & \ddots & \ddots &  \\
& & & & & T_{n-1} & 2(T_{n-2} + T_{n-1})
\end{bmatrix}\begin{bmatrix}
\dot q_1\\\vdots\\\dot  q_k\\\vdots \\ \dot q_{n-1}
\end{bmatrix}=\begin{bmatrix}
c_0-T_1\dot q_0\\\vdots\\ c_k\\\vdots \\ c_{n-2}-T_{n-2}\dot q_n
\end{bmatrix}
\end{equation*}

where $T_k=t_{k+1}-t_k$ and:

\begin{equation*}
c_k=3\frac{T_{k+1}}{T_k}(q_{k+1}-q_k)+3\frac{T_{k}}{T_{k+1}}(q_{k+2}-q_{k+1})
\end{equation*}

Thomas' algorithm is as follows:

\begin{minipage}{0.5\textwidth}
Forward elimination:
\begin{center}
\begin{algorithmic}
    \For{$k=2:1:n$} 
        \State {$m$ $\gets$ {$\frac{a_k}{b_{k-1}}$}}
        \State{$b_k$ $\gets$ {$b_k-mc_{k-1}$}}
        \State{$d_k$ $\gets$ {$d_k-md_{k-1}$}}
    \EndFor
\end{algorithmic}
\end{center}
\end{minipage}
\begin{minipage}{0.5\textwidth}
Backward substitution:
\begin{center}
\begin{algorithmic}
\State{$x_n$ $\gets$ $\frac{d_n}{b_n}$}
    \For{$k=2:1:n$} 
        \State{$x_k$ $\gets$ {$\frac{d_k-c_kx_{k+1}}{b_k}$}}
    \EndFor
\end{algorithmic}
\end{center}
\end{minipage}


\begin{figure}[H]
\centering
\includegraphics[keepaspectratio,width=\textwidth]{cubic_3}
\caption{Trajectory interpolation with continuous accelerations.}
\end{figure}

\section{Assignment 4}

\subsection{Image analysis using morphological operators}

2D image analysis can be carried out with the use of morphological operators. Image closing and opening let us remove the background of pictures and isolate the objects within them. More complex operations can then be carried out, inferring for example geometric information on the shapes produced.

\subsubsection{Example 1}

Background removed by opening with a structuring element in the shape of a disk of size 65. Shapes improved by image closing with a structuring element in the shape of a disk of size 25 and by filling holes. Noise removed after image binarization with area opening, removing all shapes with fewer than 50 pixels.

Classification used the 'Circularity' property to distinguish between coins and USB stick, as well as the 'MajorAxis' and 'MinorAxis' properties, which were averaged to compute the radius of the coins to detect which was smaller and which was bigger.

\begin{figure}[h]
\centering
\begin{minipage}{0.45\textwidth}
\includegraphics[keepaspectratio,width=0.9\textwidth]{4_coins_original}
\end{minipage}
\begin{minipage}{0.45\textwidth}
\includegraphics[keepaspectratio,width=0.9\textwidth]{4_coins_shapes}
\end{minipage}
\caption{Original image (left) vs extracted shapes (right).}
\end{figure}
\begin{figure}[h]
\centering
\includegraphics[keepaspectratio,width=0.5\textwidth]{4_coins_labels}
\caption{Labeled image.}
\end{figure}

\newpage

\subsubsection{Example 2}

Background removed by opening with a structuring element in the shape of a disk of size 65. Shapes improved by image closing with a structuring element in the shape of a disk of size 25 and by filling holes. Noise removed after image binarization with area opening, removing all shapes with fewer than 50 pixels.

Classification used the 'Circularity' property to distinguish between washers and bolts. The 'MajorAxis' and 'MinorAxis' properties were averaged to compute the outer radius of the washers, on which a quintic polynomial was fit to determine the relation with the actual inner diameter. For the bolts, the 'Orientation' property was used to align the shapes vertically, so that the bottom part of them, which contains the threaded part, could be extracted. On the extracted shapes, the 'MajorAxis' and 'MinorAxis' properties were extracted and used to fit cubic polynomials to determine, respectively, the bolt length and diameter.

\begin{figure}[h]
\centering
\begin{minipage}{0.3\textwidth}
\includegraphics[keepaspectratio,width=0.9\textwidth]{4_nuts_original}
\end{minipage}
\begin{minipage}{0.3\textwidth}
\includegraphics[keepaspectratio,width=0.9\textwidth]{4_nuts_shapes}
\end{minipage}
\begin{minipage}{0.3\textwidth}
\includegraphics[keepaspectratio,width=0.9\textwidth]{4_nuts_labels}
\end{minipage}
\caption{Original image (left) vs extracted shapes (middle) vs labels (right).}
\end{figure}

\section{Assignment 5}

\subsection{Compute the 3D trajectory (position, velocity, acceleration and jerk) in the picture as
a combination of linear and circular motion primitives and compare it with the
trajectory obtained using one of the multi-point methods.}

\begin{figure}[h]
\centering
\includegraphics[keepaspectratio,width=0.35\textwidth]{3Dtraj_ref}
\caption{3D trajectory - reference.}
\end{figure}

Operational space trajectories can be computed by composing multiple motion primitives. The motion primitives used here are:

\begin{itemize}
\item Rectilinear path:
\begin{equation*}
p(u)=p_i+u(p_f-p_i)\;\;\;\;\;u\in[0,1]
\end{equation*}
\item Circular path:
\begin{equation*}
p(u)=c+Rp'(u)=c+R\begin{bmatrix}
\rho\cos(u)\\\rho\sin(u)\\
0
\end{bmatrix}=c+\begin{bmatrix}
(P-c)' & e_3'\cross(P-c)' & e_3'
\end{bmatrix}\begin{bmatrix}
\rho\cos(u)\\\rho\sin(u)\\
0
\end{bmatrix}\;\;\;\;\;u\in[0,\theta]
\end{equation*}

where $c$ is the position of the center of the circular path, $P$ is the position of the starting point on the circle and $e_3=\begin{bmatrix}
0 & 0 & 1
\end{bmatrix}$.
\end{itemize}

For comparison, a 3D trajectory is computed by computing the smoothing cubic splines along the three directions. The smoothing spline trajectory is also computed with added waypoints to improve the tracking of the position.

\begin{figure}[H]
\centering
\includegraphics[keepaspectratio,width=0.4\textwidth]{3Dtraj}
\caption{3D trajectory - motion primitives vs smoothing splines ($w_k=\infty\forall k$, $\mu=1$).}
\end{figure}

\begin{figure}[h]
\centering
\includegraphics[keepaspectratio,width=0.6\textwidth]{3Dtraj_pos}
\caption{3D trajectory - Position comparison.}
\end{figure}

\begin{figure}[h]
\centering
\includegraphics[keepaspectratio,width=0.9\textwidth]{3Dtraj_profiles}
\caption{3D trajectory - Velocity, acceleration, jerk comparison.}
\end{figure}

\section{Assignment 6}

\subsection{Let $p_1 , p_2 , p_3$ be three points on a sphere of center $P_0$ and radius $R$. Design the trajectory such that (1) the EE will pass through the three points along the shortest path, and (2) the z axis of the EE is always orthogonal to the sphere.}

The shortest path $\gamma$ on a spherical surface lies on a great circle, which is the biggest circle that belongs to the spherical surface. Such a path is called geodesic.

First of all, the radius vectors of the starting and ending points, $p_1$ and $p_2$, of the geodesic are used to compute its direction:

\begin{equation*}
\begin{matrix}
r_1 = \frac{(p_1-p_0)}{R}\\
r_2 = \frac{(p_2-p_0)}{R}
\end{matrix}\implies r_g = r_1\cross r_2
\end{equation*}

The rotation matrix that rotates the great circle with $z=0$ to align it with the geodesic is then:

\begin{equation*}
R_\gamma = \begin{bmatrix}
r_1 & r_g\cross r_1 & r_g
\end{bmatrix}
\end{equation*}

where the cross product between $r_g$ and $r_1$ was arbitrarily chosen ($r_2$ would have worked as well) to obtain a third orthogonal direction.

Using the rotation matrix we can rotate the parametric form of the great circle as follows:

\begin{equation*}
\gamma = R_\gamma\begin{bmatrix}
R\cos(u)\\ R\sin(u)\\ 0
\end{bmatrix} + p_0
\end{equation*}

From the parametrization, we obtain the axes of the Frenet frame associated to each point:

\begin{equation*}
t = \frac{d\gamma}{du} = R_\gamma\begin{bmatrix}
-R\sin(u)\\ R\cos(u)\\ 0
\end{bmatrix}\;\;\;\;\;\;\;n = \frac{d^2\gamma}{du^2} = R_\gamma\begin{bmatrix}
-R\cos(u)\\ -R\sin(u)\\ 0
\end{bmatrix}\;\;\;\;\;\;\;b = t\cross n
\end{equation*}

The computed normal vector $n$ is always perpendicular to the sphere surface. In fact it lays on the same direction of the corresponding radius vector of the parametric geodesic:

\begin{equation*}
r_\gamma = \frac{\gamma-p_0}{R}= R_\gamma\begin{bmatrix}
\cos(u)\\ \sin(u)\\ 0
\end{bmatrix}\implies r_\gamma\cross n = R_\gamma\left(\begin{bmatrix}
\cos(u)\\ \sin(u)\\ 0
\end{bmatrix}\cross\begin{bmatrix}
-\cos(u)\\ -\sin(u)\\ 0
\end{bmatrix}\right)
\end{equation*}

The cross product in the parentheses is null for any value of $u$, therefore $n$ is always parallel to the corresponding radius vector. Since the radius vector's direction is perpendicular to the sphere by definition and the normal vector $n$ starts from a point belonging to the sphere's surface, the vector $n$ is always perpendicular to the sphere's surface.

\begin{figure}[h]
\centering
\includegraphics[keepaspectratio,width=0.8\textwidth]{sphere}
\caption{Geodesics between three random sphere points with the associated Frenet frames ($n$ in blue, $t$ in green, $b$ in red).}
\end{figure}

\section{Assignment 7}

\subsection{Plan the pick-and-place task for the UR5 robot in ROS.}

The pick-and-place task is divided into two phases: scanning and pick-and-place.

The scanning consists of a composition of rectilinear and circular motion primitives that make sure that the camera mounted on the end-effector of the robot covers the whole workspace, detecting all cubes and their destination.

\begin{figure}[h]
\centering
\includegraphics[keepaspectratio,width=0.9\textwidth]{scanning}
\caption{End-effector path and associated Frenet frames for the scanning phase.}
\end{figure}

After the cubes and their destinations have been found, the pick-and-place phase is carried out as follows:

\begin{itemize}
\item Move linearly in Cartesian space towards the cube location.
\item Lower the end-effector onto the cube, close the gripper and then raise the end-effector back up.
\item Move linearly to the home position.
\item Move linearly to the destination.
\item Lower the end-effector, open the gripper and then raise the end-effector back up.
\item Move linearly to the home position.
\end{itemize}

\chapter{Vision}

\section{Assignment 1}

\subsection{Implement in MATLAB 3rd-, 5th-, 7th-order polynomials for $q_i>q_f$ and $q_i<q_f$ and for $t\in [t_i,t_f]$ and $t\in[0,\Delta T]$}

All polynomial trajectories can be expressed as:

\begin{align*}
q(t) &= a_7(t-t_i)^7 + a_6(t-t_i)^6 + a_5(t-t_i)^5 + a_4(t-t_i)^4 + a_3(t-t_i)^3 + a_2(t-t_i)^2 + a_1(t-t_i) + a_0\\
\dot q(t) &= 7a_7(t-t_i)^6 + 6a_6(t-t_i)^5 + 5a_5(t-t_i)^4 + 4a_4(t-t_i)^3 + 3a_3(t-t_i)^2 + 2a_2(t-t_i) + a_1\\
\ddot q(t) &= 42a_7(t-t_i)^5 + 30a_6(t-t_i)^4+20a_5(t-t_i)^3 + 12a_4(t-t_i)^2 + 6a_3(t-t_i) + 2a_2\\
\dddot q(t) &= 210a_7(t-t_i)^4 + 120a_6(t-t_i)^3+60a_5(t-t_i)^2 + 24a_4(t-t_i) + 6a_3\\
\ddddot q(t) &= 840a_7(t-t_i)^3 + 360a_6(t-t_i)^2+120a_5(t-t_i) + 24a_4
\end{align*}

For the 3rd-order polynomial $a_7=a_6=a_5=a_4=0$ while for the 5th-order polynomial $a_7=a_6=0$.

The problem of determining the $a_i$ coefficients of the polynomials is solved by setting up a system of equations using initial and final conditions on velocity (3rd-order), velocity and acceleration (5th-order), velocity, acceleration and jerk (7th-order).

\begin{figure}[H]
\centering
\includegraphics[keepaspectratio,width=\textwidth]{poly_1}
\caption{3rd-, 5th-, 7th-order polynomial trajectories with $q_i<q_f$ and $t\in[0,\Delta T]$, $q_i=0,q_f=1,\Delta T=1, v_i=v_f=a_i=a_f=j_i=j_f=0$.}
\label{fig:poly_1}
\end{figure}

\begin{figure}[H]
\centering
\includegraphics[keepaspectratio,width=\textwidth]{poly_2}
\caption{3rd-, 5th-, 7th-order polynomial trajectories with $q_i>q_f$ and $t\in[0,\Delta T]$, $q_i=2,q_f=1,\Delta T=1, v_i=v_f=a_i=a_f=j_i=j_f=0$.}
\label{fig:poly_2}
\end{figure}

\begin{figure}[H]
\centering
\includegraphics[keepaspectratio,width=\textwidth]{poly_3}
\caption{3rd-, 5th-, 7th-order polynomial trajectories with $q_i<q_f$ and $t\in[t_i,t_f]$, $q_i=0,q_f=1,t_i=1,t_f=2, v_i=v_f=a_i=a_f=j_i=j_f=0$.}
\label{fig:poly_3}
\end{figure}

\begin{figure}[H]
\centering
\includegraphics[keepaspectratio,width=\textwidth]{poly_4}
\caption{3rd-, 5th-, 7th-order polynomial trajectories with $q_i>q_f$ and $t\in[t_i,t_f]$, $q_i=2,q_f=1,t_i=1,t_f=2, v_i=v_f=a_i=a_f=j_i=j_f=0$.}
\label{fig:poly_4}
\end{figure}

\section{Assignment 2}

\subsection{Implement in MATLAB the trapezoidal trajectory taking into account the different constraints.}

The general expression for a trapezoidal velocity profile is:

\begin{equation*}
q(t)=\begin{cases}
q_i+\dot q_i(t-t_i)+\frac{\dot q_c-\dot q_i}{2t_a}(t-t_i)^2 & t_i\leq t\leq t_a+t_i\\
q_i+\dot q_i\frac{t_a}{2}+\dot q_c(t-t_i-\frac{t_a}{2})^2 & t_a+t_i\leq t\leq t_f-t_d\\
q_f-\dot q_f(t_f-t)-\frac{\dot q_c-\dot q_f}{2t_d}(t_f-t)^2 & t_f-t_d\leq t\leq t_f
\end{cases}
\end{equation*}

If the initial and final velocities are null, then $t_a=t_d=t_c$.

If $\dot q_i=\dot q_f=0$ then the possible constraints are:
\begin{itemize}
\item $t_c$:
\begin{equation*}
\ddot q_c = \frac{q_f-q_i}{t_ct_f-t_c^2}\;\;\;\;\dot q_c=\ddot q_ct_c
\end{equation*}
\item $\ddot q_c$:
\begin{equation*}
t_c=\frac{t_f}{2}-\frac{1}{2}\sqrt{\frac{t_f^2\ddot q_c-4(q_f-q_i)}{\ddot q_c}}\;\;\;\;\dot q_c=\ddot q_ct_c
\end{equation*}
\item $\dot q_c$:
\begin{equation*}
t_c=\frac{q_i-q_f+\dot q_ct_f}{\dot q_c}\;\;\;\;\ddot q_c = \frac{\dot q_c^2}{q_i-q_f+\dot q_ct_f}
\end{equation*}
\item $\ddot q_c,\dot q_c$:
\begin{equation*}
t_c=\frac{\dot q_c}{\ddot q_c}\;\;\;\;t_f=\frac{\dot q_c^2+\ddot q_c(q_f-q_i)}{\dot q_c\ddot q_c}
\end{equation*}
\end{itemize}

In all cases the feasibility condition is that $2t_c\leq t_f-t_i$.

If $\dot q_i,\dot q_f\neq 0$ the possible constraints are:

\begin{itemize}
\item $\ddot q_{c,max}$:
\begin{equation*}
\dot q_c = \frac{1}{2}(\dot q_i+\dot q_f+\ddot q_{c,max}\Delta T+\sqrt{\ddot q_{c,max}^2\Delta T^2-4\ddot q_{c,max}\Delta q+2\ddot q_{c,max}(\dot q_i+\dot q_f)\Delta T-(\dot q_i-\dot q_f)^2}\;\;\;\;\;t_a=\frac{\dot q_c-\dot q_i}{\ddot q_{c,max}}\;t_d=\frac{\dot q_c-\dot q_f}{\ddot q_{c,max}}
\end{equation*}

The trajectory is feasible when the argument of the square root is positive, and when the maximum acceleration satisfies:

\begin{equation*}
\ddot q_{c,max}\Delta q>\frac{\abs{\dot q_i^2-\dot q_f^2}}{2}\;\;\;\ddot q_{c,max}\geq\ddot q_{c,lim}=\frac{2\Delta q-(\dot q_i-\dot q_f)\Delta T+\sqrt{4\Delta q^2-4\Delta q(\dot q_i+\dot q_f)\Delta T+2(\dot q_i^2+\dot q_f^2)\Delta T^2}}{\Delta T^2}
\end{equation*}

when $\ddot q_{c,max}=\ddot q_{c,lim}$ there is no constant velocity phase.
\item $\ddot q_{max},\dot q_{max}$:
First we compute the condition:
\begin{equation*}
\ddot q_{c,max}\Delta q\gtreqless\dot q_{c,max}^2-\frac{\dot q_i^2+\dot q_f^2}{2}
\end{equation*}

If the above is $>$ then:

\begin{equation*}
\dot q_c=\dot q_{c,max}\;\;\;t_a=\frac{\dot q_{c,max}-\dot q_i}{\ddot q_{c,max}}\;\;\;t_d=\frac{\dot q_{c,max}-\dot q_f}{\ddot q_{c,max}}\;\;\;\Delta T=\frac{\Delta q\ddot q_{c,max}+\dot q_{c,max}^2}{\ddot q_{c,max}\dot q_{c,max}}
\end{equation*}

If the above is $\leq$ then:
\begin{equation*}
\dot q_c=\dot q_{c,lim}=\sqrt{\ddot q_{c,max}\Delta q+\frac{\dot q_i^2+\dot q_f^2}{2}}<\dot q_{c,max}\;\;\;t_a=\frac{\dot q_{c,lim}-\dot q_i}{\ddot q_{c,max}}\;\;\;t_d=\frac{\dot q_{c,lim}-\dot q_f}{\ddot q_{c,max}}
\end{equation*}
\end{itemize}

\begin{figure}
\begin{minipage}{0.333\textwidth}
\centering
\includegraphics[keepaspectratio,width=\textwidth]{trap_1}
\caption{$t_c=2s$}
\label{fig:trap_1}
\end{minipage}
\begin{minipage}{0.333\textwidth}
\centering
\includegraphics[keepaspectratio,width=\textwidth]{trap_2}
\caption{$t_c=2s, q_i>q_f$}
\label{fig:trap_2}
\end{minipage}
\begin{minipage}{0.333\textwidth}
\centering
\includegraphics[keepaspectratio,width=\textwidth]{trap_3}
\caption{$t_c=2s, t_i\neq 0$}
\label{fig:trap_3}
\end{minipage}
\end{figure}

\begin{figure}
\begin{minipage}{0.5\textwidth}
\centering
\includegraphics[keepaspectratio,width=\textwidth]{trap_5}
\caption{$v_c=1.5$}
\label{fig:trap_5}
\end{minipage}
\begin{minipage}{0.5\textwidth}
\centering
\includegraphics[keepaspectratio,width=\textwidth]{trap_4}
\caption{$v_c=v_{c,lim}=2$}
\label{fig:trap_4}
\end{minipage}
\end{figure}

\begin{figure}
\begin{minipage}{0.5\textwidth}
\centering
\includegraphics[keepaspectratio,width=\textwidth]{trap_6}
\caption{$a_c=2$}
\label{fig:trap_6}
\end{minipage}
\begin{minipage}{0.5\textwidth}
\centering
\includegraphics[keepaspectratio,width=\textwidth]{trap_7}
\caption{$a_c,v_c=2$}
\label{fig:trap_7}
\end{minipage}
\end{figure}

\begin{figure}
\begin{minipage}{0.5\textwidth}
\centering
\includegraphics[keepaspectratio,width=\textwidth]{trap_8}
\caption{$\dot q_i=1,\dot q_f=0.5,\ddot q_{max}=0.5$}
\label{fig:trap_8}
\end{minipage}
\begin{minipage}{0.5\textwidth}
\centering
\includegraphics[keepaspectratio,width=\textwidth]{trap_9}
\caption{$\dot q_i=1,\dot q_f=0.5,\dot q_{max}=1.5,\ddot q_{max}=0.5$}
\label{fig:trap_9}
\end{minipage}
\end{figure}

To generate a multipoint trajectory we simply repeat the computation for each consecutive point pair. To improve the resulting velocity and acceleration profiles we can apply an heuristic to assign velocities to each waypoint, so that a less jagged profile is obtained:

\begin{align*}
\dot q(t_i)&=\dot q_i\\
\dot q(t_k)&=\begin{cases}
0 & \text{if }sign(\Delta Q_k)\neq sign(\Delta Q_{k+1})\\
sign(\Delta Q_k)\dot q_{max}& \text{if }sign(\Delta Q_k)= sign(\Delta Q_{k+1})\\
\end{cases}\\
\dot q(t_f)&=\dot q_f
\end{align*}

\begin{figure}
\begin{minipage}{0.5\textwidth}
\centering
\includegraphics[keepaspectratio,width=\textwidth]{trap_10}
\caption{Multipoint trajectory without velocity heuristic}
\label{fig:trap_10}
\end{minipage}
\begin{minipage}{0.5\textwidth}
\centering
\includegraphics[keepaspectratio,width=\textwidth]{trap_11}
\caption{Multipoint trajectory with velocity heuristic}
\label{fig:trap_11}
\end{minipage}
\end{figure}

\section{Assignment 3}
\subsection{Interpolating polynomials with computed velocities at path points and imposed velocity at initial/final points.}

To implement multipoint trajectories we concatenate cubic splines. The interpolating trajectory is then:

\begin{equation*}
q(t)\coloneqq\{\Pi_k(t),t\in[t_k,t_{k+1}],k=0,\dots,n-1\}
\end{equation*}

where:

\begin{equation*}
\Pi(t)=a_3^k(t-t_k)^3+a_2^k(t-t_k)^2+a_1^k(t-t_k)+a_0^k
\end{equation*}

Fixing velocities on all path points yields:

\begin{equation*}
\begin{cases}
a_0^k&=q_k\\
a_1^k&=\dot q_k\\
a_2^k&=\frac{1}{T_k}\left(\frac{3(q_{k+1}-q_k)}{T_k}-2\dot q_k-\dot q_{k+1}\right)\\
a_3^k&=\frac{1}{T_k^2}\left(\frac{^2(q_{k}-q_{k+1})}{T_k}+\dot q_k+\dot q_{k+1}\right)\\
\end{cases}
\end{equation*}

where $T_k=t_{k+1}-t_k$.

\begin{figure}[h]
\centering
\includegraphics[keepaspectratio,width=\textwidth]{cubic_1}
\caption{Trajectory through $q_k=\begin{bmatrix}
10 & 20 & 30 & 0 & 40
\end{bmatrix}$ at times $t_k=\begin{bmatrix}
0 & 2 & 4 & 8 & 10
\end{bmatrix}$ with \\velocities $\dot q_k=\begin{bmatrix}
0 & 0 & 0 & 5.2 & 0
\end{bmatrix}$}
\end{figure}

Path point velocities are not generally known. We can estimate them with Euler's approximation:

\begin{equation*}
v_k = \frac{q_k-q_{k-1}}{t_k-t_{k-1}}\implies\begin{cases}
\dot q(t_0)&=\dot q_0\\
\dot q(t_k)&=\begin{cases}
0 & \text{if }sign(v_k)\neq sign(v_{k+1})\\
\frac{v_k+v_{k+1}}{2} &  \text{if }sign(v_k) =  sign(v_{k+1})\\
\end{cases}\\
\dot q(t_n) &= \dot q_n
\end{cases}
\end{equation*}

\begin{figure}[h]
\centering
\includegraphics[keepaspectratio,width=\textwidth]{cubic_2}
\caption{Trajectory interpolation with Euler's approximation.}
\end{figure}

\newpage

\subsection{Interpolating polynomials with continuous accelerations at path points and imposed velocity at initial/final points (+ Thomas algorithm)}


Imposing velocity and acceleration continuity at path points results in the definition of a linear system $A\dot q=c$, where $A$ is a tridiagonal matrix. Thanks to this property the system can be solved for $\dot q$ efficiently using Thomas' algorithm. Given:

\begin{equation*}
\begin{bmatrix}
2(T_0+T_1) & T_0 &  &\\
T_2 & 2(T_1+T_2) & T_1 &  & \\
 & \ddots & \ddots & \ddots &  & \\
& & T_{k+1} & 2(T_k+T_{k+1}) & T_k &  & \\
& & & \ddots & \ddots & \ddots &  \\
& & & & & T_{n-1} & 2(T_{n-2} + T_{n-1})
\end{bmatrix}\begin{bmatrix}
\dot q_1\\\vdots\\\dot  q_k\\\vdots \\ \dot q_{n-1}
\end{bmatrix}=\begin{bmatrix}
c_0-T_1\dot q_0\\\vdots\\ c_k\\\vdots \\ c_{n-2}-T_{n-2}\dot q_n
\end{bmatrix}
\end{equation*}

where $T_k=t_{k+1}-t_k$ and:

\begin{equation*}
c_k=3\frac{T_{k+1}}{T_k}(q_{k+1}-q_k)+3\frac{T_{k}}{T_{k+1}}(q_{k+2}-q_{k+1})
\end{equation*}

Thomas' algorithm is as follows:

\begin{minipage}{0.5\textwidth}
Forward elimination:
\begin{center}
\begin{algorithmic}
    \For{$k=2:1:n$} 
        \State {$m$ $\gets$ {$\frac{a_k}{b_{k-1}}$}}
        \State{$b_k$ $\gets$ {$b_k-mc_{k-1}$}}
        \State{$d_k$ $\gets$ {$d_k-md_{k-1}$}}
    \EndFor
\end{algorithmic}
\end{center}
\end{minipage}
\begin{minipage}{0.5\textwidth}
Backward substitution:
\begin{center}
\begin{algorithmic}
\State{$x_n$ $\gets$ $\frac{d_n}{b_n}$}
    \For{$k=2:1:n$} 
        \State{$x_k$ $\gets$ {$\frac{d_k-c_kx_{k+1}}{b_k}$}}
    \EndFor
\end{algorithmic}
\end{center}
\end{minipage}


\begin{figure}[H]
\centering
\includegraphics[keepaspectratio,width=\textwidth]{cubic_3}
\caption{Trajectory interpolation with continuous accelerations.}
\end{figure}

\section{Assignment 4}

\subsection{Image analysis using morphological operators}

2D image analysis can be carried out with the use of morphological operators. Image closing and opening let us remove the background of pictures and isolate the objects within them. More complex operations can then be carried out, inferring for example geometric information on the shapes produced.

\subsubsection{Example 1}

Background removed by opening with a structuring element in the shape of a disk of size 65. Shapes improved by image closing with a structuring element in the shape of a disk of size 25 and by filling holes. Noise removed after image binarization with area opening, removing all shapes with fewer than 50 pixels.

Classification used the 'Circularity' property to distinguish between coins and USB stick, as well as the 'MajorAxis' and 'MinorAxis' properties, which were averaged to compute the radius of the coins to detect which was smaller and which was bigger.

\begin{figure}[h]
\centering
\begin{minipage}{0.45\textwidth}
\includegraphics[keepaspectratio,width=0.9\textwidth]{4_coins_original}
\end{minipage}
\begin{minipage}{0.45\textwidth}
\includegraphics[keepaspectratio,width=0.9\textwidth]{4_coins_shapes}
\end{minipage}
\caption{Original image (left) vs extracted shapes (right).}
\end{figure}
\begin{figure}[h]
\centering
\includegraphics[keepaspectratio,width=0.5\textwidth]{4_coins_labels}
\caption{Labeled image.}
\end{figure}

\newpage

\subsubsection{Example 2}

Background removed by opening with a structuring element in the shape of a disk of size 65. Shapes improved by image closing with a structuring element in the shape of a disk of size 25 and by filling holes. Noise removed after image binarization with area opening, removing all shapes with fewer than 50 pixels.

Classification used the 'Circularity' property to distinguish between washers and bolts. The 'MajorAxis' and 'MinorAxis' properties were averaged to compute the outer radius of the washers, on which a quintic polynomial was fit to determine the relation with the actual inner diameter. For the bolts, the 'Orientation' property was used to align the shapes vertically, so that the bottom part of them, which contains the threaded part, could be extracted. On the extracted shapes, the 'MajorAxis' and 'MinorAxis' properties were extracted and used to fit cubic polynomials to determine, respectively, the bolt length and diameter.

\begin{figure}[h]
\centering
\begin{minipage}{0.3\textwidth}
\includegraphics[keepaspectratio,width=0.9\textwidth]{4_nuts_original}
\end{minipage}
\begin{minipage}{0.3\textwidth}
\includegraphics[keepaspectratio,width=0.9\textwidth]{4_nuts_shapes}
\end{minipage}
\begin{minipage}{0.3\textwidth}
\includegraphics[keepaspectratio,width=0.9\textwidth]{4_nuts_labels}
\end{minipage}
\caption{Original image (left) vs extracted shapes (middle) vs labels (right).}
\end{figure}

\section{Assignment 5}

\subsection{Compute the 3D trajectory (position, velocity, acceleration and jerk) in the picture as
a combination of linear and circular motion primitives and compare it with the
trajectory obtained using one of the multi-point methods.}

\begin{figure}[h]
\centering
\includegraphics[keepaspectratio,width=0.35\textwidth]{3Dtraj_ref}
\caption{3D trajectory - reference.}
\end{figure}

Operational space trajectories can be computed by composing multiple motion primitives. The motion primitives used here are:

\begin{itemize}
\item Rectilinear path:
\begin{equation*}
p(u)=p_i+u(p_f-p_i)\;\;\;\;\;u\in[0,1]
\end{equation*}
\item Circular path:
\begin{equation*}
p(u)=c+Rp'(u)=c+R\begin{bmatrix}
\rho\cos(u)\\\rho\sin(u)\\
0
\end{bmatrix}=c+\begin{bmatrix}
(P-c)' & e_3'\cross(P-c)' & e_3'
\end{bmatrix}\begin{bmatrix}
\rho\cos(u)\\\rho\sin(u)\\
0
\end{bmatrix}\;\;\;\;\;u\in[0,\theta]
\end{equation*}

where $c$ is the position of the center of the circular path, $P$ is the position of the starting point on the circle and $e_3=\begin{bmatrix}
0 & 0 & 1
\end{bmatrix}$.
\end{itemize}

For comparison, a 3D trajectory is computed by computing the smoothing cubic splines along the three directions. The smoothing spline trajectory is also computed with added waypoints to improve the tracking of the position.

\begin{figure}[H]
\centering
\includegraphics[keepaspectratio,width=0.4\textwidth]{3Dtraj}
\caption{3D trajectory - motion primitives vs smoothing splines ($w_k=\infty\forall k$, $\mu=1$).}
\end{figure}

\begin{figure}[h]
\centering
\includegraphics[keepaspectratio,width=0.6\textwidth]{3Dtraj_pos}
\caption{3D trajectory - Position comparison.}
\end{figure}

\begin{figure}[h]
\centering
\includegraphics[keepaspectratio,width=0.9\textwidth]{3Dtraj_profiles}
\caption{3D trajectory - Velocity, acceleration, jerk comparison.}
\end{figure}

\section{Assignment 6}

\subsection{Let $p_1 , p_2 , p_3$ be three points on a sphere of center $P_0$ and radius $R$. Design the trajectory such that (1) the EE will pass through the three points along the shortest path, and (2) the z axis of the EE is always orthogonal to the sphere.}

The shortest path $\gamma$ on a spherical surface lies on a great circle, which is the biggest circle that belongs to the spherical surface. Such a path is called geodesic.

First of all, the radius vectors of the starting and ending points, $p_1$ and $p_2$, of the geodesic are used to compute its direction:

\begin{equation*}
\begin{matrix}
r_1 = \frac{(p_1-p_0)}{R}\\
r_2 = \frac{(p_2-p_0)}{R}
\end{matrix}\implies r_g = r_1\cross r_2
\end{equation*}

The rotation matrix that rotates the great circle with $z=0$ to align it with the geodesic is then:

\begin{equation*}
R_\gamma = \begin{bmatrix}
r_1 & r_g\cross r_1 & r_g
\end{bmatrix}
\end{equation*}

where the cross product between $r_g$ and $r_1$ was arbitrarily chosen ($r_2$ would have worked as well) to obtain a third orthogonal direction.

Using the rotation matrix we can rotate the parametric form of the great circle as follows:

\begin{equation*}
\gamma = R_\gamma\begin{bmatrix}
R\cos(u)\\ R\sin(u)\\ 0
\end{bmatrix} + p_0
\end{equation*}

From the parametrization, we obtain the axes of the Frenet frame associated to each point:

\begin{equation*}
t = \frac{d\gamma}{du} = R_\gamma\begin{bmatrix}
-R\sin(u)\\ R\cos(u)\\ 0
\end{bmatrix}\;\;\;\;\;\;\;n = \frac{d^2\gamma}{du^2} = R_\gamma\begin{bmatrix}
-R\cos(u)\\ -R\sin(u)\\ 0
\end{bmatrix}\;\;\;\;\;\;\;b = t\cross n
\end{equation*}

The computed normal vector $n$ is always perpendicular to the sphere surface. In fact it lays on the same direction of the corresponding radius vector of the parametric geodesic:

\begin{equation*}
r_\gamma = \frac{\gamma-p_0}{R}= R_\gamma\begin{bmatrix}
\cos(u)\\ \sin(u)\\ 0
\end{bmatrix}\implies r_\gamma\cross n = R_\gamma\left(\begin{bmatrix}
\cos(u)\\ \sin(u)\\ 0
\end{bmatrix}\cross\begin{bmatrix}
-\cos(u)\\ -\sin(u)\\ 0
\end{bmatrix}\right)
\end{equation*}

The cross product in the parentheses is null for any value of $u$, therefore $n$ is always parallel to the corresponding radius vector. Since the radius vector's direction is perpendicular to the sphere by definition and the normal vector $n$ starts from a point belonging to the sphere's surface, the vector $n$ is always perpendicular to the sphere's surface.

\begin{figure}[h]
\centering
\includegraphics[keepaspectratio,width=0.8\textwidth]{sphere}
\caption{Geodesics between three random sphere points with the associated Frenet frames ($n$ in blue, $t$ in green, $b$ in red).}
\end{figure}

\end{document}